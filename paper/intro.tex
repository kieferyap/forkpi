\section{Introduction}

Modern keyless door security systems are still largely inaccessible, primarily due to their high cost \cite{KeylessDoorProsCons}. As a result, such systems are currently confined in the business setting. Moreover, most door security systems are proprietary black boxes, meaning their code is closed to the public, and their components are non-removable and non-interchangeable. Without knowing what is under the hood, end users cannot tweak the code to suit their private needs, and should a component fail, they are left with no choice but to replace the entire unit.

The goal of this study is to create a door security solution that is more accessible to the public, by lowering the acquisition cost, releasing the code publicly, and making the individual components customizable. Anyone should be able to replicate the entire system simply by assembling the components, and running our code.

There are two main components to our system. SpoonPi is the component that is installed onto doors; it is responsible for allowing or denying users access. ForkPi is the component where you register new users; it is responsible for maintaining the database that all SpoonPis will access. There are as many SpoonPis as there are doors to be secured, but only one ForkPi for the entire network. To save on resources, the single ForkPi unit can also function as a SpoonPi unit, although there are some drawbacks to this set-up which will be discussed later on.

Our system supports five authentication mechanisms: single-factor RFID, single-factor fingerprint, RFID \& PIN, fingerprint \& PIN, and RFID \& fingerprint. There are no system-wide constraints with regards to the mechanism to be used. Users can choose to authenticate using any combination they wish; SpoonPi can handle all five mechanisms by default.

%% TODO: Discuss results