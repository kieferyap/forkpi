\section{Results and Discussion}

\subsection{Implementation}
Single-factor authentication was implemented only for RFID, since single-factor PIN was deemed to be unacceptable. The system waits for an RFID tag to be swiped, and if the RFID tag's UID is present in the database, the user is allowed entry for five seconds before the system locks the door once again.

The two-factor authentication process starts off by asking the user for an RFID input. If the UID of the RFID tag is present in the database, the system then prompts the user for their 4-digit PIN. If the RFID-PIN combination is in the database, then the user is authorized.

If the user presents an unrecognized RFID tag, then the system will immediately deny access without prompting for a PIN. If the user presents a valid RFID tag but enters an incorrect PIN, the system will also deny access. If the user consecutively enters the wrong PIN for an RFID tag five times, the tag will be denied access for thirty minutes. When being prompted for a PIN, the user must enter his/her PIN with no more than ten seconds between key presses.

\subsection{Security Metrics}

\subsubsection{PIN}
Attackers using brute force would need to input, on average, half of all possible permutations before guessing the correct PIN.

For 4-digit PIN, that would be $ g = \frac12 \cdot 10^4 = 5,000 $ guesses. Plugging in $t=5s, L=1800s, n=5$, it would take, on average, $506.94$ hours or about $21$ days to crack.

For 6-digit PIN, that would be $ g = \frac12 \cdot 10^6 = 500,000 $ guesses. Plugging in $t=5s, L=1800s, n=5$, it would take, on average, $50694$ hours or about $5.78$ years to crack.

\subsubsection{RFID}
The issue of stolen tokens is mitigated by allowing administrators to delete compromised RFID codes from the database. With regard to the issue of replicated tokens, MiFare cards used in the system cannot have their codes reprogrammed \cite{MiFareAdafruit}, however, they can be replicated using special cloning devices\cite{Yung2013}.

\subsection{Limitations and Recommendations}
As of now, the web application ForkPi cannot scan for RFID tags at the same time as the authentication system. If ForkPi is scanning for RFID tags and the user closes the web browser, the system will crash the next time someone polls for an RFID tag. Both of these limitations are due to the RFID polling being unable to handle more than one pending request.

Another limitation is that the fingerprint component still does not work. The reason for this is that the datasheet is in Chinese and the library is built for Arduino, not Raspberry Pi.

Future researchers trying to create a similar system may choose Arduino as the target platform instead of the Raspberry Pi, since a good number of the ready-made code that work with the components are for Arduino.