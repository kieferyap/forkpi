% \newpage
\section{Appendix}
\appendix
\section{Raspberry Pi Setup}
\begin{enumerate}
    \item Installing the OS
        \begin{enumerate}[label*=\arabic*.]
            \item Download NOOBS\_vX\_X\_XX.zip from \url{http://www.raspberrypi.org/downloads/}    and send these files to your SD.
            \item The SD Card must have the following specifications:
                \begin{enumerate}[label*=\arabic*.]
                    \item At least 8GB of data
                    \item Format system: FAT32
                    \item Allocation unit size: 8192 bytes
                \end{enumerate}
            \item Insert SD Card to Raspberry Pi and boot
            \item Once the installation window shows up, click on the checkbox for "Raspbian" and hit "I" (Install)
            \item Default login information:
                \begin{enumerate}[label*=\arabic*.]
                    \item username: pi
                    \item password: raspberry
                \end{enumerate}
            \item After the installation, a menu in blue background will pop up. (To access this menu again in the future, do "sudo raspi-config") Do the following:
                \begin{enumerate}[label*=\arabic*.]
                    \item Go to Option 3: "Enable Boot to Desktop/Scratch Choose whether to boot into a desktop environment, Scratch, or the command line" and hit enter.
                    \item Select the boot option "Desktop Log in as user 'pi' at the graphical desktop."
                    \item Hit "FINISH" and reboot. Congrats! Your desktop should be all set and you're ready to go.
                \end{enumerate}
        \end{enumerate}

    \item Preliminaries
        \begin{enumerate}[label*=\arabic*.]
            \item Run the following commands in the terminal. (Make sure that you're connected to the internet first.)
                \begin{enumerate}[label*=\arabic*.]
                    \item \texttt{sudo apt-get update}
                    \item \texttt{sudo apt-get install git}
                \end{enumerate}
            
            \item Create a new folder, and navigate to that folder in the terminal.
                \begin{enumerate}[label*=\arabic*.]
                % \item \texttt{git clone https://github.com/crentagon/rbpi_thesis}
                    \item \texttt{git init}
                    \item \texttt{git remote add origin https://github.com/crentagon/rbpi_thesis}
                    \item \texttt{git pull origin master}
                \end{enumerate}
                
            \item Setting the resolution of your monitor, if need to:
                \begin{enumerate}[label*=\arabic*.]
                    % \item \texttt{git clone https://github.com/crentagon/rbpi_thesis}
                    \item \texttt{tvservice -d edid}
                    \item \texttt{edidparser edid}
                    \item Before the line "HDMI:EDID preferred mode remained as..." at the bottom of edid, select the resolution that you desire.
                    \item For example, if the resolution desired is:
                    
                    \texttt{"HDMI:EDID CEA mode (4) 1280x720p @ 60 Hz with pixel clock..."}
                    
                    \item \texttt{sudo nano /boot/config.txt}
                    
                    \item Look for the lines
                    
                    \texttt{# hdmi_group = x}
                    \texttt{# hdmi_mode = y}
                    
                    \item Take out the hashtags, and then:
                    
                        \begin{enumerate}[label*=\arabic*.]
                        \item Change x to 1 if it's CEA
                        \item Change x to 2 if it's DMT
                        \item Change y to 4 to the number in parenthesis
                        \end{enumerate}
                
                \end{enumerate}
            
            \item Changing the Keyboard Layout
                \begin{enumerate}[label*=\arabic*.]
                    \item \texttt{sudo raspi-config}
                    \item Select the fourth option and choose keyboard settings.
                    \item Reboot for the changes to take effect.
                \end{enumerate}
        \end{enumerate}

    \item Required Installation
\end{enumerate}

\section{Why ForkPi?}
The name of the web application, ForkPi, comes from the hardware components that were used for the system: \textbf{F}ingerprint, \textbf{O}LED, \textbf{R}FID, \textbf{K}eypad and the Raspberry \textbf{Pi}.

\texttt{sudo su}